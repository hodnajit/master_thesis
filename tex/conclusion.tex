\label[chap-conclusion]
\chap Conclusion and future work
Lorem ipsum sit amet

\sec Usage of the system in the industry
	As I already mentioned in Chapter \ref[chap-introduction], the key features for industrial usage of localization are
	\begitems
	* reliability of the system in the long term with keeping a number of manual interventions minimal,
	* reduce the influence of the industrial environment on a system,
	* generality of the system to be possible to deploy it on different robotic platforms,
	* precision and repeatability.
      \enditems
	
	The resulting system is designed as an Aided navigation system. The main component is the inertial navigation system, which provides short-term accurate relative location, but is affected by drift. The estimated INS error removes the drift based on the absolute localization from the UWB beacon and the linear velocity from the robot odometry.
	
	The proposed implementation reduces the influence of the structure of the vehicle or the morphology of the ground. Also, the localization principle is not based on any external landmarks or environment contours, reducing the requirement for a static environment to make the localization reliable in the long term without many manual interventions. In this point of view, the system architecture looks like it satisfies the industrial requirements quite well.

	The accuracy of UWB localization defines the accuracy of the whole system, and therefore, the discussion is directed mainly to this area. UWB localization is relatively accurate and well usable when the robot is moving. On the contrary, when standing, it is necessary to reduce the effect of UWB localization on the result. This effect can be very well ensured by the so-called virtual sensors when an artificially created signal enters the filtering at suitable moments. For example, if the vehicle is not moving, the virtual sensor produces zero velocities. This point is included in a possible future improvement of the system.
	
The accuracy of the UWB system is also reduced in No Line of Sight environments, and therefore, it is necessary to have a relatively dense beacon network so that such phenomena do not occur. The location of the beacon and the precise estimation of their positions are significant for the entire system's resulting accuracy.

	Another important aspect is the actual mounting of sensors on the robotic platform. Especiallt, the placement of the UWB tag affects the resulting precision of UWB localization. Two positions of the UWB tag on CART2 platform were evaluated during experiments. The first position is near the metal parts of the robot and 0.2 m above the ground. The second is 1.m above the robotic platform without any metal part around. The (TODO: add graph) graph shows that possible signal reflections from the ground or metal parts on the robot distort the UWB localization.

	Experiments should be performed in a larger industrial environment to evaluate the precision and repeatability of the system properly. Unfortunately, it was impossible to provide such an environment because installing anchors themselves and providing an external localization system is costly. However, the accuracy and repeatability of the localization are illustrated in the Figure (TODO: add graph), where it can be seen that similar results are obtained with repeated passes from the same starting position and with the same control displays.
	
	As shown in this section, this system meets the latest requirements for use in industry, and therefore it would be appropriate to test it in more realistic scenarios directly in the industrial environment. The

\sec Conclusion
This thesis aims to design an indoor real-time localization system for autonomous ground vehicles based on Ultra wide-band technology.

The resulting system's architecture is based on an Inertial navigation system aided with a measurement from absolute localization given by the UWB localization system and odometry of the AGV. Results show a significant decrement of the positioning error compared to the UWB localization and reliable attitude estimation without significant drift. The system was evaluated following a standardized testing method, considering the horizontal position error and the yaw angle as the primary performance metrics.

\secc Jak byly splnene jednotlive zadani prace

\secc Jak byl system otestovan a jake jsou shrnujici vysledky

\secc Shrnuti zaverem


\sec Future work
Future work on the proposed localization system can be divided into two main categories.
\begitems
* Possible improvements of the algorithm itself and hardware specifics
* and testing the system in more complex scenarios.
\enditems

\secc Possible improvements of the system
	The first possible improvements are aimed at using {\bf UWB localization in RTDoA mode}. This mode offers a higher update rate and, at the same time, the ability to use the system on multiple AGVs at once, as mentioned in section \ref[UWB].	
	
	Another option is to use {\bf virtual sensors} that improve the precision of the entire system and are commonly used in similar applications (TODO: Add citation).
	
	For a long-term reliable localization system, it is also possible to design {\bf calibration processes}. The calibration enables obtaining the most precise data from sensors and is valid for sensors such as IMU or odometry. Today, calibrations are typically applied before the entire system is switched on and during the ride. Thus, the next step for improving the reliability of the system is to design calibration processes.
	
	The final product of the REX project, which also includes this thesis, is {\bf the localization of AMR in 2D and 3D}. So further improvements aim in this direction. It is necessary to experimentally verify whether the proposed system works for localization on flying drones and, if necessary, to suggest possible improvements to enable this localization.

\secc Experimental verification in complex testing scenario
The initial experiments evaluated in this work look promising. The next step is to verify the system in a more complex test environment. The test scenarios should include
	\begitems
	* testing the system on {\it multiple AGVs at once}, 
	* verifying functionality {\it in an industrial environment} on the demand of industrial partners,
	* finding out the possible influence of {\bf moving people in the environment} (the human body could be a significant obstacle for UWB signal).
	\enditems