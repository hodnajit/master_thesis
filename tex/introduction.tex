\label[chap-introduction]
\chap Introduction
	Indoor real-time localization is a crucial component in autonomous mobile robotics, and nowadays, the interest for precise localization is growing due to the fourth industrial revolution influencing all industries. There are existing approaches and technologies to deal with indoor localization in the industry but do not fully meet all of the requirements of the fourth industrial revolution.
	
	One of the requirements is to keep as few manual interventions as possible for the technology to work reliably for a long period of time. An example is autonomous ground vehicles (AGV) manufactured by the company Ceit in Škoda factories, which are localized by continuous magnetic tapes physically mounted on the factory’s ground. The AGV follows these magnetic tapes, which often can be damaged by moving AGVs or people and can not be easily replaced or modified.
	
	Another possible requirement is the modularity of the factory environment. Today’s trend is to create a factory that consists of modular parts, and the entire production process is assembled according to its current needs. Therefore, approaches based on localization relying heavily on landmarks or contours from cameras or lidars may fail. These approaches must address long-term sustainability, and this topic is not straightforward and can lead to difficulties.
	
	The demand is also for a localization that will be universally applicable to various robotic platforms, whether ground vehicles of different shapes and constructions or flying drones. These requirements are currently being met in the outdoor environment by the Global Navigation Satellite System (GNSS), which unfortunately is not suitable for use in indoor environments. The use of external beacons seems to be a reasonable solution because it has similar properties as GNSS. These beacons can be based, for example, on ultrasound or radio waves as ultra wide-band or Wi-Fi.

	A single technology cannot meet all of these requirements, but an appropriate fusion of carefully chosen approaches can.
	
\sec Aims and requirements
	This thesis is assigned by the Czech company Datavision s. r. o. and is a part of a project called Guidance and Localization upgrade creating Autonomous Mobile Robots. The abbreviation for this project is REX, and it is also used in this thesis.
	
	
	REX aims to create fleet management of autonomous mobile robots, including their localization, control, navigation and planning. The project is co-financed by the Technology Agency of the Czech Republic (TACR) under the TREND Programm FW03010020 and aims to satisfy the fourth industrial revolution requirements. 
	
	
	This thesis aims to propose an indoor real-time localization system, which includes both the position and orientation of AGV. This aim is closely connected with the specification of localization in the REX system but is simplified into 2D. The fundamentals need to be applicable in 3D with a few modifications.
	
	The essential aim is to design a localization system based on ultra wide-band technology and onboard dead reckoning sensors, which should improve the UWB localization itself.
	
\sec Structure of the thesis
This thesis is organized into six major parts. The following Chapter \ref[chap-sensors] describes the fundamentals of the sensors used and the concept of inertial navigation systems. Chapter \ref[chap-state-est] discusses existing data fusion algorithms for pose estimation algorithms and their pros and cons. The proposed localization and its implementation are described in Chapter \ref[chap-rloc]. In Chapter \ref[chap-experiments], the proposed localization is experimentally evaluated in two environments. The thesis is concluded in Chapter \ref[chap-conclusion], where the usage of this system in the industrial environment is given, followed by a summary of the thesis output and a few proposals for improving and extending the work.
