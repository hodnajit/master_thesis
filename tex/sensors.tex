\label[chap-sensors]
\chap Sensors
In chapter \ref[chap-sensors] the overview of used sensors and their properties is given. The main aim is to describe an ultra-wideband technology(UWB), an inertial measurement unit (IMU) and an odometry.

\label[IMU]
\sec Inertial measurement unit
An inertial measurement unit (IMU) is a device that utilizes measurement systems such as gyroscopes and accelerometers to estimate the relative position, velocity and acceleration of a vehicle in motion \cite[handbook_robotics]. The sensor is typically integrated with an on-board computational unit and could contain more sensors as a magnetometer or thermometer.

The gyroscopes measure changes its angular velocities and accelerometers a specific forces, which can be easily transformed into linear accelerations \cite[handbook_robotics]. The IMU typically contains three orthogonal accelerometers and three orthogonal gyroscopes. Because of that, it can measure angular velocities and specific force in each axis to maintain a 6-DOF estimate of the pose of the vehicle (position ($x$, $y$, $z$) and orientation ($roll$, $pitch$, $yaw$). The process of the computation can be seen in the figure \ref[fig-imu-diag].
\midinsert
\picw=\hsize \cinspic imgs/sensors/imu.pdf
\label[fig-imu-diag]
\caption/f IMU block diagram \cite[handbook_robotics]
\endinsert

There are two basic inertial navigation systems (also called mechanization architectures) \cite[handbook_robotics]\cite[handbook_gnss_ins] how the IMU is mounted to a vehicle.
\begitems
* In {\bf gimbaled systems}, the IMU is attached to a stabilized platform that maintains its inertial orientation as the vehicle manoeuvres.
* In {\bf strap-down systems}, it is rigidly attached to the vehicle.
\enditems
The mechanization determines which navigation equations are used during the estimation of linear accelerations and angular velocities of the vehicle.

IMU's are extremely sensitive to measurement errors given by properties of used gyroscopes, accelerometers and their mounting. As the data are once or twice integrated, any error in measurement causes a linear or quadratic error in the pose estimation. Even with a small measurement error, over time, the IMU's drift becomes significant, and it needs to be externally compensated. The IMU's provides a short-term stable solution because gyroscopes and accelerations have relatively low noise characteristics in a short period of time, and it has a high data rate (100 Hz - 200 Hz).


\secc Accelerometers
Accelerometers can measure external forces acting on the vehicle. They measure a specific force relatively to a non-rotating inertial space in a specific direction. They are sensitive to all forces, including gravity and fictitious forces \cite[handbook_robotics]. These forces need to be compensated during transformation to a local-level navigation frame that is not inertial.

{\bf Mechanical accelerometers} use a spring-mass-damper system. The force acts on the mass, and it causes displacement of the spring. The system is limited by, in reality, non-ideal spring and sensitivity to vibration.


{\bf Microelectromechanical systems based accelerometers (MEMS)} are made of at least three components, namely a proof mass, a suspension to hold the mass and a pickoff, which relates an output signal to the induced accelerations \cite[mems_navigation]. MEMS accelerometers are then classified by the type of converting the mechanical displacement of the proof mass to an electrical signal. In most common principles belong to piezoresistive, capacitive sensing, piezoelectric, optical sensing and tunnelling current sensing. The piezoelectric MEMS sensors can not be used for navigation because their output rate is too low \cite[mems_navigation].

\secc Gyroscopes
Gyroscopes are used for estimating a rotational motion of a body, each gyroscope measures angular rate $\omega$ (inertial angular rotation) relatively to a non-rotating inertial space in one axis. There are basically three main categories of gyroscopes \cite[handbook_robotics].

{\bf Mechanical gyroscopes} have a mass spinning steadily with respect to a free movable axis, they are not used a lot anymore, but they can be found in very old submarines.

{\bf Optical gyroscopes} are based on the Sagnac effect, which states that frequency/phase shift between two waves counter-propagating in a rotating ring interferometer is proportional to the loop angular velocity. As a light source, the laser is typically used. Currently, this technology gives the best performance. Examples can be ring laser gyroscopes (RLG) or fibre optic gyroscopes (FOG).

{\bf Vibrating gyroscopes} are based on the Coriolis effect that induces a coupling between two resonant modes of a mechanical resonator. Example can be MEMS sensors \cite[mems_navigation], these are typically the cheapest, and they can be found basically everywhere, for example, in every mobile device. As MEMS gyroscopes have no rotating parts, have low power consumptions requirements, and are small, they replaced others in robotics applications.

The performace and application of each technology is perfectly demonstrate in figure \ref[fig-gyro].
\midinsert
\picw=\hsize \cinspic imgs/sensors/gyr_usage.jpg
\label[fig-gyro]
\caption/f Gyroscopes technology plotted by size and performance \cite[epson_gyro].
\endinsert


\secc The characteristics of IMU, Allan variance analysis
	{\bf Allan variance(AVAR)} is widely used to characterize the performance of inertial sensors in time-domain. The AVAR $\sigma_A^2(\tau)$ is a function of the averaging time $\tau$, computed as
	$$	\sigma_A^2(\tau) = {1 \over 2(N-1)} \sum_{i=1}^{N-1}(\overline{y}_{\tau}(i+1)-\overline{y}_{\tau}(i))^2, \eqmark $$
	where $N$ represents the number of clusters in the dataset ($N=floor(M/n)$), $n$ is the number of samples in the cluster, $M$ is the total number of samples in dataset, $\tau$ is the time length of the cluster ($m \times T_s$), $T_s$ is the sampling period, $\overline{y}_{\tau}(i+1)$ and $\overline{y}_{\tau}(i))$ are mean value of certain cluster corresponding to $i$ \cite[signal_processing].
	
	The confidence of the deviation of current cluster is given by
	$$	\delta_{\sigma}(\tau) = {1 \over \sqrt{2({M \over n}-1)}}\eqmark$$ and for getting more confidence, the ovelapping AVAR is used. That means, that the samples in the cluster overlapped.
	
	The process of measuring AVAR consist of collecting 24-48 hours long dataset of static table test when the inertial sensor is not moving, and it is in not vibrating environments (no trains, subways which caused vibration).
	
	If the dataset is valid and the AVAR is correctly computed, the plot copies the example plot seen in figure \ref[avar_scheme]. It is typically plotted on a log/log scale. The plot can be divided into specific parts, where the slope separates all error sources. The most important implication for navigation purposes is the fact that until the bias instability (slope=0) is reached, the sensor model can contain only a white (gaussian) noise \cite[white_noise]. By that time, the external reset needs to be done.

	
	{\bf The error characteristic for IMU}
	 

\midinsert
\picw=\hsize \cinspic imgs/sensors/avar_scheme.png
\label[avar_scheme]
\caption/f An example of Allan variance plot \cite[avar_scheme]
\endinsert

\secc Grades of IMUs according to their application


\secc Navigation equations






