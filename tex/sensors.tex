\label[chap-sensors]
\chap Sensors
	In Chapter \ref[chap-sensors], an overview of the sensors used and their properties is given. The main aim is to describe localization methods based on ultra-wideband technology(UWB), an inertial measurement unit (IMU) followed by inertial navigation systems (INS), and odometry.
	
	The section dedicated to UWB localization briefly introduces the UWB signals follows with three most used localization techniques: Two-Way Ranging, Time Difference of Arrival, and Reverse Time Difference of Arrival.
	
	The IMU section mainly focuses on the unit overview, a short description of gyroscopes and accelerometers used in it, the errors of these sensors, its analysis and outcomes for the localization unit. This section is closely related to the INS section, where principles of INS are explained and a strap-down implementation example is given.
	
	The last section includes basic information about odometry and its benefits as well as which odometry is used during the experimental verification of the proposed localization system.
\label[UWB]
\sec Localization based on ultra-wideband 
{\bf Ultra-wideband} is an emerging wireless personal area network (PAN) radio technology with a wide range of uses. The most promising usage in the field of robotics is accurate indoor localization because its benefits include
\begitems
* high data rates,
* high time resolution,
* low power consumption,
* multipath immunity,
* low costs,
* small size
* and simultaneous ranging communication \cite[uwb_history].
\enditems
The UWB signal is defined as a signal with an absolute bandwidth ($B$) of at least 500 MHz, defined as

$$	B = f_H -f_L, \eqmark$$
where $f_H$ is the upper frequency and $f_L$ is the lower frequency,
or with a fractional bandwidth of larger than 20 \% given by
$$	B_{frac} = {B \over f_c} = {B \over {f_H + f_L \over 2}} \cite[uwb_positioning_system_book]. \eqmark $$

Since the bandwidth of the signal is wide, the power spectral density is low. It means that even if the UWB signals share a spectrum with some narrowband signals such as WiFi, it essentially behaves as environmental noise and does not affect any other narrowband signals much \cite[uwb_positioning_system_book]. Also, since the length of each pulse is small, the possibility of overlapping the original pulse is reduced. Thus, it should be robust against multipath problem \cite[uwb_positioning_system_book].

For precise communication, a direct line of sight should be established between the transmitter and the receiver. However, as UWB signals consist of many frequencies, some of them can reflect well off of some objects, while others can penetrate through them \cite[uwb_positioning_system_book].

A few {\bf localization techniques exist based on UWB} signals which are exchanged between a tag and several reference anchors with known positions \cite[uwb_positioning_system_book]. The more accurate results in the line of sight environments are based on measuring the signal’s time of flight from several devices, namely Two-Way Ranging (TWR), Time Difference of Arrival (TDoA) and Reverse Time Difference of Arrival (RTDoA). The overview of these technologies follows. 

{\bf Two-Way ranging} (TWR) is a simple method where the tag and anchors exchanges message in both ways. Thus, the updated rate of the tag’s position is limited and decreased with a higher number of tags asking for their positions. The synchronization of the messages is provided with one anchor declared as an initiator of the network. The communication between the tag and anchors is illustrated in Figure \ref[fig-twr]. This method is the most used nowadays, as it is the simplest \cite[uwb_positioning_system_book].

\midinsert
\picw=0.5\hsize \cinspic imgs/sensors/twr.jpg
\clabel[fig-twr]{Two way ranging communication}
\caption/f Two way ranging communication \cite[twr].
\endinsert

In {\bf Time Difference of Arrival} (TDoA) technology, the tag only transmits information and the anchors only receive data. The tag sends a message to all available anchors. They estimate the time difference of the delivered messages, and the tag’s position is calculated accordingly. With this technology, a higher update rate can be achieved even with more tags than TWR \cite[uwb_positioning_system_book].

In {\bf Reverse Time Difference of Arrival} (RTDoA), the tag listens and anchors transmit into the environment. The position is directly computed in the tag. This technology has no upper limit for tags and also promises the highest update rates. Thus, it is well suited for flying drones or for similar applications where this high update rate and low latency are needed \cite[uwb_positioning_system_book]. This mode is comparable to the localization algorithm used in GNSS. Unfortunately, this technology is not yet commonly available on the market.

This article \cite[ins_gnss_uwb] proves the similarity of the use of GNSS and UWB.  The authors demonstrate the possibility of switching from GNSS to UWB while a pedestrian user moves from an outdoor space to an indoor environment.

Despite all of the mentioned benefits, in practice, UWB localization faces errors caused by surrounding factors, leading to coordinate jitters or outlying results \cite[performance_enhancement].


\label[IMU]
\sec Inertial measurement unit
An inertial measurement unit (IMU) is a device that utilizes measurement systems such as gyroscopes and accelerometers to estimate the relative position, velocity and acceleration of a vehicle in motion \cite[handbook_robotics]. The unit is typically integrated with an onboard computational system and may contain more sensors acting as a magnetometer or thermometer.

The gyroscopes measure angular velocities and accelerometers specific forces which can be easily transformed into linear accelerations \cite[handbook_robotics]. The IMU typically contains three orthogonal accelerometers and three orthogonal gyroscopes. Because of that, it can measure angular velocities and specific forces in each axis to maintain a 6-DOF estimate of the pose of the vehicle (position ($x$, $y$, $z$) and orientation ($roll$, $pitch$, $yaw$). The process of the computation can be seen in Figure \ref[fig-imu-diag].
\midinsert
\picw=\hsize \cinspic imgs/sensors/imu.pdf
\clabel[fig-imu-diag]{IMU block diagram.}
\caption/f IMU block diagram \cite[handbook_robotics].
\endinsert

There are two basic ways to mount the IMU to a vehicle, also called mechanization architectures \cite[handbook_robotics, handbook_gnss_ins].
\begitems
* In {\bf gimbaled systems}, the IMU is attached to a stabilized platform that maintains its inertial orientation as the vehicle maneuvers.
* In {\bf strap-down systems}, it is rigidly attached to the vehicle.
\enditems
The mechanization determines the conversion between measurements of IMU and estimation of linear accelerations and angular velocities of the vehicle. It means the transformation of the IMU body frame to the local frame. The conversion is closely related to inertial navigation systems described in Section \ref[INS].

IMU's are extremely sensitive to measurement errors. The way in which the gyroscopes and accelerometers are mounted, along with their properties, play a major factor in how accurate their results are. As the data are once or twice integrated, any error in measurement causes a linear or quadratic error in the pose estimation. Even with a small measurement error, the IMU's drift becomes significant, and it needs to be externally compensated. The IMU provides a short-term stable solution, which is not affected by the external environment \cite[handbook_gnss_ins], and it has a high data rate (100 Hz - 200 Hz). That makes the IMU measurement complementary to the UWB localization measurement.

IMU's are extremely sensitive to measurement errors given by properties of used gyroscopes, accelerometers and their mounting. As the data are once or twice integrated, any error in measurement causes a linear or quadratic error in the pose estimation. Even with a small measurement error, the IMU’s drift becomes significant and it needs to be externally compensated for. The IMU provides a short-term stable solution, which is not affected by the external environment \cite[handbook_gnss_ins] and it has a high data rate (100 Hz - 200 Hz). This is what makes the IMU measurement complementary to the UWB localization measurement.

\label[acc]
\secc Accelerometers
Accelerometers can measure external forces acting on the vehicle. They measure a specific force relatively to a non-rotating inertial space in a specific direction. They are sensitive to all forces, including gravity and fictitious forces \cite[handbook_robotics].

{\bf Mechanical accelerometers} use a spring-mass-damper system. When a force acts on 
the mass, it causes a displacement of the spring. The system is limited by the physical properties of the actual spring.


{\bf Microelectromechanical systems (MEMS) based accelerometers} are made of at least three components, namely a proof mass, a suspension to hold the mass and a pickoff, which relays an output signal to the induced accelerations \cite[mems_navigation]. MEMS accelerometers are then classified by converting the mechanical displacement of the proof mass to an electrical signal. The most common principles belong to piezoresistive, capacitive sensing, piezoelectric, optical sensing and tunneling current sensing. Unlike the others, the piezoelectric MEMS sensors can not be used for navigation because their output rate is too low \cite[mems_navigation].

The current accelerometers use technology according to an application that is summarized in Figure \ref[fig-acc].

\midinsert
\picw=\hsize \cinspic imgs/sensors/acc_usage.png
\clabel[fig-acc]{Accelerometers technology plotted by bias instability and scale factor stability.}
\caption/f Accelerometers technology plotted by bias instability and scale factor stability \cite[acc_tech].
\endinsert

\label[gyr]
\secc Gyroscopes
Gyroscopes are used for estimating a rotational motion of a body. Each gyroscope measures angular rate $\omega$ (inertial angular rotation) relatively to a non-rotating inertial space in one axis. There are three main categories of gyroscopes \cite[handbook_robotics].

{\bf Mechanical gyroscopes} have a mass spinning steadily with respect to a free movable axis, they are not often used anymore, but they can be found in very old submarines.

{\bf Optical gyroscopes} are based on the Sagnac effect, which states that frequency/phase shift between two waves counter-propagating in a rotating ring interferometer is proportional to the loop angular velocity. As a light source, a laser is typically used. Currently, this technology gives the best performance. Examples can be ring laser gyroscopes (RLG) or fibre optic gyroscopes (FOG).

{\bf Vibrating gyroscopes} are based on the Coriolis effect that induces a coupling between two resonant modes of a mechanical resonator. Typically, vibrating gyroscopes are based on MEMS technology \cite[mems_navigation], and they play a significant role in robotics because of their simplicity. They are small, cheap, have no rotating parts and have low power consumption.

The performance and application of each technology are demonstrated in Figure \ref[fig-gyro].
\midinsert
\picw=\hsize \cinspic imgs/sensors/gyr_usage.png
\clabel[fig-gyro]{Gyroscopes technology plotted by bias instability and scale factor stability.}
\caption/f Gyroscopes technology plotted by bias instability and scale factor stability \cite[acc_tech].
\endinsert

\label[chap-sec-avar]	
\secc IMU's errors and Allan variance analysis
	{\bf IMU errors}
IMUs faces several error sources which are always related to the specific sample unit and its technology. In this thesis, the main focus is given to MEMS-based IMUs as they are used in the experiments. These sensors are typically small and inexpensive. This section summarises the most significant errors for MEMS sensors and analyzes these errors with the Allan variance. This analysis is applied to a specific IMU which is subsequently used during the experiments.
	
	
	Errors can be divided into two categories \cite[mems_navigation]
	\begitems
		* stochastic errors, which can be described as random processes,
		* and systematic errors, which are caused by manufacturing imperfections or handling issues with IMU. These errors can be corrected by proper calibration.
	\enditems
	Nevertheless, errors need to be analyzed and reduced according to the application’s requirements. The next following subset of errors is the most significant according to the topic of this thesis.
	
	{\em Biases} of accelerometers and gyroscopes used in IMU are examples of systematic errors and can be divided into
	
		\begitems
		* bias instability\fnote{also called in-run bias}, which represents drift of the sensor over time,
		* and initial bias\fnote{repeatability bias} a static offset, which can vary when starting up the device each time, but during the run, it remains static.
		\enditems
		Biases are typically represented in $^\circ /hr$ or $^\circ /s$ for gyroscopes and $mg$ for accelerometers.
		
	{\em A scale factor} and {\em a misalignment error}, both systematic errors, could also prove to be significant. The scale factor is connected to imperfections while converting the real measurement input value and output value. The nonorthogonality of all sensors gives the misalignment error in IMU, and it is caused during production.
		
	{\em Angle} or {\em velocity random walks} belong to stochastic errors. The measurement of gyroscopes and accelerometers is subject to white noise (the noise represented by Gaussian distribution). During the estimation of angles and velocities, integration needs to be done. Then the white noise starts to manifest itself by angle or velocity random walk, $(^\circ /s/\sqrt{Hz})$ and $(m^2/s/\sqrt{Hz})$ respectively.

	{\bf Allan variance(AVAR)} is widely used to analyze a random error of inertial sensors in the time domain and is the most common time-domain measure of frequency stability. AVAR's brief introduction and important outcomes are given in \cite[avar].
	
	
	The AVAR $\sigma_A^2(\tau)$ is a function of the averaging time $\tau$, computed as
	$$	\sigma_A^2(\tau) = {1 \over 2(N-1)} \sum_{i=1}^{N-1}(\overline{y}_{\tau}(i+1)-\overline{y}_{\tau}(i))^2, \eqmark $$
	where $N$ represents the number of clusters in the dataset ($N=floor(M/n)$), $n$ is the number of samples in the cluster, $M$ is the total number of samples in the dataset, $\tau$ is the time length of the cluster ($\tau = m \times T_s$), $T_s$ is the sampling period, $\overline{y}_{\tau}(i+1)$ and $\overline{y}_{\tau}(i))$ are mean values of certain cluster of $i+1$-th and $i$-th cluster respectively \cite[signal_processing].
	
	\midinsert
    \picw=0.6\hsize \cinspic imgs/sensors/avar.png
    \clabel[fig-avar]{The difference between non-overlapping and overlapping samples.}
    \caption/f The difference between non-overlapping and overlapping sample \cite[epson_g365].
	\endinsert
	
The samples in a cluster can be both non-overlapping and overlapping. The difference is illustrated in \ref[fig-avar]. The overlapping samples improve the confidence of the resulting estimate. That is why this method is generally the most common for measuring time-domain frequency stability \cite[avar].
	

	The process of measuring AVAR consist of collecting 24-48 hour long datasets when the inertial sensor is not moving and the environment is not vibrating (no trains or subways that would cause vibration). The sampling values are angular rates or accelerations.
	
	If the dataset is valid and the AVAR is correctly computed, the plot copies the example plot seen in Figure \ref[fig-avar_scheme]. It is typically plotted on a log/log scale. A different slope of the graph describes each noise component by that the chart can be easily divided into specific parts.
	
	\midinsert
	\picw=0.9\hsize \cinspic imgs/sensors/avar_scheme.png
	\clabel[fig-avar_scheme]{An example of Allan variance plot.}
	\caption/f An example of Allan variance plot \cite[avar_scheme].
	\endinsert
	
	The most significant outcome for navigation is when the bias instability is reached (slope is zero). At this time, the sensor model contains only a white (Gaussian) noise \cite[white_noise]. After that period, an external reset needs to be done.	 

\secc Performance of IMUs according to their application
	IMUS can be used in various applications, which differs by IMUs performance. Figure \ref[imu_perform] summarises the overview of each sensor's precision for a given application is summarized in Figure \ref[imu_perform].
	
	\midinsert
	\picw=0.6\hsize \cinspic imgs/sensors/imu_performance.png
	\clabel[imu_perform]{A performance of IMU per application.}
	\caption/f A performance of IMU per application \cite[imu_performance].
	\endinsert
	
\label[INS]
\sec Inertial navigation systems
The fundamental idea behind Inertial navigation systems (INS) is integrating a linear acceleration into a position. Because of that, this topic is closely connected with the IMU, as it measures current linear acceleration \cite[ins_95]. The integration of IMU measurement is given by {\bf navigation equations}.

Since INS is typically used for navigating aircraft, the principle of INS is introduced in this example. The plane is moving in a navigation coordinate frame. This frame can be specified as
\begitems
*a local-level frame (as North-East-Down or East-North-Up),
* as a reference to a specific point on planet Earth
* or an Earth-fixed frame as ECEF \cite[mems_navigation],
and these frames are demonstrated in Figure \ref[fig-navigate-frames].

\midinsert
\line{\hsize=.5\hsize \vtop{%
      \clabel[fig-navigate-frames]{Example of coordinate frames used in INS}
      \picw=\hsize \cinspic imgs/sensors/coordinate_systems.pdf
      \caption/f Example of coordinate frames used in INS\fnote{The picture is taken from Lecture 11 of Aircraft Avionics course taught at CTU FEE in Prague.}.
   \par}\vtop{%
      \clabel[fig-body_sensor]{Example of body and sensor frame}
      \picw=\hsize \cinspic imgs/sensors/body_frame.pdf
      \caption/f Example of body and sensor frame.
   \par}}
\endinsert

An IMU is mounted on an aircraft and its gyroscopes and accelerometers measure in its sensor frame. The aircraft's frame is called the body frame, so the measurement needs to be transformed into the body frame (see Figure \ref[fig-body_sensor]). The output of the INS is given in the navigation frame. Thus, the last transformation is from the body frame into a navigation frame.
		
The navigation equations describe exactly what the transformation is between the sensor and navigation frames as well as the integration of IMU measurements. For example, navigation signal processing in strap-down INS can be seen in Figure \ref[strapdown].

\midinsert
	\picw=0.7\hsize \cinspic imgs/sensors/strapdown.png
	\clabel[strapdown]{Schema of strapdown INS.}
	\caption/f Schema of strapdown INS \cite[gnss_ins_integration].
\endinsert
	
		
\label[odometry]
\sec Odometry
Odometry is an example of a dead reckoning system. It estimates the pose and velocity of a device based on internal relative measurements of its motion. It can be obtained from various sources as IMU, lidars, cameras or wheel encoders \cite[handbook_robotics].

Both IMU and wheel encoders are used in the suggested localization system since they can counter each other's negative characteristics since wheel encoders drift over traveled distance and IMU drift over time \cite[handbook_robotics]. In this thesis, the term odometry is used to refer to wheel encoders unless stated otherwise.

The details of odometry information vary with vehicle design, and during experiments, the differential type is used (illustrated in Figure \ref[diff_drive]).

	\midinsert
	\picw=0.5\hsize \cinspic imgs/sensors/differential_drive.png
	\clabel[diff_drive]{A differential drive kinematics scheme.}
	\caption/f A differential drive kinematics scheme \cite[handbook_robotics].
	\endinsert









