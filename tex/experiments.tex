\label[chap-experiments]
\chap Experiments
This chapter is dedicated to the experimental evaluation of the proposed localization system. Firstly, the description of evaluation approaches is given. As the system is tested in real experiments, the second section is dedicated to implementing the localization system on the robotic platform CART2. This section contains a brief introduction to specific IMU, UWB network and wheel encoders, and the robotic platform itself.

The following two sections are dedicated to experiments done in two different experimental environments. The first set of experiments, first described in the third section, were realized in a lab at Datavision s.r.o. company\fnote{To find out more about the company follow link: \url{https://datavision.software/}} with an external localization system based on an AprilTag detection with a camera. The external localization system was developed in this thesis for the evaluation of the proposed localization system.

Additional experiments took place in an Intelligent and mobile robotics lab at the Czech Technical University in Prague - Czech Institute of Informatics, Robotics, and Cybernetics, where the Vicon reference system is used for the evaluation. These experiments are summarized in the fourth section of this chapter.

\sec Description of the evaluation
First, it is necessary to determine which variables exhibit the performance of the system well. Then choose the appropriate test scenarios on which to evaluate the behaviour of the individual variables. It is good to select a suitable reference system and perform the entire evaluation on a set of predefined metrics.

The evaluation is focused on four variables, which are
\begitems
* position x,
* position y
* point in 2D space (xy)
* and angle of rotation.
\enditems

It is worth noting that the output of the proposed system is the position in a 3D space. However, this work aimed to design a localization in a 2D space, and therefore an accent is put on these four states. For system evaluations, it is advisable to have an external reference system, which will provide us with reference data with which we can compare the results of the evaluated system. The data from the reference and evaluated systems must have unified timestamps to compare their outputs easily.

Experimental scenarios are chosen so that their difficulty increases. The first tests are performed in a smaller area. Experiments start with a stationary test and simple movements at shorter distances (moving in one axis, rotating an one spot). Then, more complex trajectories are approached, such as the following rectangular and infinity shape trajectories. Further experiments are performed in a larger space where more complex trajectories are tested.

Data from both reference and evaluated systems are then processed as follows. Each reference measurement's timestamp is taken according to which the nearest measurement can be found in the evaluated data. This filtering is vital for reference systems that have a lower frequency than the proposed localization.

The criteria that interest us are
\begitems
* the evolution of individual variables over time,
* the visualization of the trajectories from both systems
* and the errors of variables versus the reference.
\enditems

These metrics are used to analyze whether the proposed system converges or diverges from reality and whether the system is subject to drift. The analysis of errors according to the root mean square, average, median, minimum and maximum calculation defines the system's resulting accuracy and precision.

All these testing scenarios and metrics are further applied in two test environments.

\sec Used hardware description
This section presents an overview of used sensors for the proposed localization and a description of the CART2 robotic platform used during experiments.
\secc Sensors specification
The onboard sensors of interest are the Inertial Measurement Unit, ultra-wideband localization tag, April tag and encoders on motors.

The {\bf inertial measurement unit} used during the experiments is the Epson M-G365PDF1 (loaner sample). The Epson M-G365 is used in various applications ranging from stabilization systems (as a camera gimbal) to navigation systems.

\midinsert
\line{\hsize=.5\hsize \vtop{%
      \clabel[fig-epson-g365]{IMU used in experiments Epson M-G365PDF1}
      \picw=0.5\hsize \cinspic imgs/experiments/g365.jpg
      \caption/f IMU used in experiments Epson M-G365PDF1.
   \par}\vtop{%
      \clabel[fig-epson-avar]{Experimental setup for static data acquisition of Epson M-G365PDF1}
      \picw=\hsize \cinspic imgs/experiments/avar_epson_setup.png
      \caption/f Experimental setup for static data acquisition of Epson M-G365PDF1.
   \par}}
\endinsert

The IMU has six degrees of freedom and measures angular rates and linear accelerations in three axes. It is factory calibrated and the calibration data are stored in the memory of the unit. Technical specifications of the Epson M-G365 can be found at \fnote{Datasheet of Epson M-G365 \url{https://global.epson.com/products_and_drivers/sensing_system/download_hidden/pdf/m-g365pd_datasheet_e_rev20201007.pdf}}, while the summary is included in Table \ref[spec_epson]

\midinsert \clabel[spec_epson]{Technical specifications of Epson M-G365PDF1}
\ctable{lr}{
\hfil
Specification & Value \crl \tskip4pt
Triple gyroscopes & $\pm$ 450 °/sec \cr
Gyroscopes bias instability & 1.2 °/hr \cr
Gyroscopes initial bias error & 0.1 °/s \cr
Angular random walk & 0.08 °/√hr \cr
Tri-axis accelerometers & $\pm$ 10 G \cr
Accelerometers bias instability & 16 $\mu$G \cr
Accelerometers initial bias error & 3 mG \cr
Velocity random walk & 0.033 (m/s)/√hr \cr
}
\caption/t Technical specifications of Epson M-G365PDF1\cite[epson_g365].
\endinsert

As I already mentioned in Chapter \ref[chap-sec-avar], the AVAR analysis of IMU sensors can give us a brief overview of IMU's specifications. The experimental setup for static data acquisition can be seen in Figure \ref[fig-epson-avar]. The sensor is mounted on two sponges and fixed with cardboard. The static data were recorded for 48 hours at a frequency of 30.0 Hz. The experiment took place in a village without any subways, trams or trains to reduce the external vibrations (to reduce potential outliers) and at standard room temperature (about 23 °C).

For the AVAR computation, I used a python library named AllanTools. For the purposes of this thesis, the overlapping Allan deviation function is used.

\picw=.6\hsize
\centerline {\inspic imgs/experiments/epson_accelerometers.pdf \hfil \inspic imgs/experiments/epson_gyroscopes.pdf }\nobreak
\centerline {a)\hfil b)}\nobreak\medskip
\clabel[fig-epson-avar2]{Overlapping Allan variance plot for Epson M-G365PDF1.}
\caption/f Overlapping Allan variance plot for Epson M-G365PDF1.

As shown in Figure \ref[fig-epson-avar2], the external reset of integration of IMU measurement must be performed at 0.01 Hz at the latest. After that period, the non-gaussian noise from the accelerometers must be taking into account. The UWB localization system works at 10 Hz, which should be enough to reset.

The IMU is set up for the final experiments to publish the delta angle and the delta velocity at 100 Hz with a moving average filter with tap 64. Because of that, higher frequencies above 10 Hz are filtered, as shown in Figure \ref[fig-epson-filter].

\midinsert
    \picw=0.9\hsize \cinspic imgs/experiments/epson_filter.png
    \clabel[fig-epson-filter]{Moving average filter characteristics for Epson M-G365PDF1.}
    \caption/f Moving average filter characteristics for Epson M-G365PDF1\cite[epson_g365].
\endinsert

The {\bf UWB localization system} is provided by Qorvo's MDEK1001 ultra-wideband development kit. This kit includes 12 DWM1001-DEV development boards completely enclosed in plastic, see Figure \ref[fig-uwb-dwm].

\midinsert
    \picw=0.4\hsize \cinspic imgs/experiments/mdek1001.jpg
    \clabel[fig-uwb-dwm]{DWM1001-DEV development boards.}
    \caption/f DWM1001-DEV development boards\cite[mdek_1001].
\endinsert

Each can be configured as an anchor, tag or bridge node. The system is installed with six fixed anchors that are mounted and one tag which can move. The anchors are higher than the tag. According to the initiator anchor, the tag's position gives the coordinates frame's origin. The setup of the system can be seen in Figure \ref[fig-uwb-anchors_tags].

\midinsert
    \picw=0.8\hsize \cinspic imgs/experiments/uwb_anchors_tags.png
    \clabel[fig-uwb-anchors_tags]{Positioning of anchors and tags.}
    \caption/f Positioning of anchors and tags\cite[mdek_1001].
\endinsert

The anchors should all be mounted at the same height and higher than the operating area of the moving tag. They also should not be mounted close to any metal to get the best accuracy possible. There is a mobile application available to configure the network. The positions of the anchors are estimated manually and set in the network configuration.

Technical specifications of the MDEK1001 and the DWM1001-DEV can be found in documents \cite[mdek_1001], the summary is listed in Table \ref[spec_mdek1001].

\midinsert \clabel[spec_mdek1001]{System performance of MDEK1001}
\ctable{lr}{
\hfil
Specification & Value \crl \tskip4pt
Localization technology & Two-way ranging \cr
Maximum tag location rate & 10 Hz \cr
X-Y location accuracy & < 10 cm \cr
Point to point range & up to 60 m in line of sight conditions \cr
Scheme range & 25 - 30 m between anchors\cr
}
\caption/t System performance of MDEK1001\cite[mdek_1001].
\endinsert

{\bf The odometry} is computed according to the measurement of encoders in Maxon EPOS4 positioning controllers for Maxon brushless DC motors.

{\bf AprilTag} serves as a global reference for the pose of the CART2 platform. The AprilTag detection software computes the precise 3D position, orientation, and identity of the tags relative to the camera\cite[april_tag]. This tag is similar to QR codes (a type of two-dimensional bar code), but it encodes smaller data payloads (between 4-12 bits), and it can be detected more robustly.

The {\bf camera} used to detect the AprilTag was the Niceboy Stream Pro with Full HD (1920 x 1080) resolution, 30 FPS, 90 ° field of vision and a f/1.8 lens aperture\cite[camera_niceboy].

\midinsert
\line{\hsize=.5\hsize \vtop{%
      \clabel[fig-apriltag]{AprilTag used for detection of CART2 position}
      \picw=4cm \cinspic imgs/experiments/apriltag.png
      \caption/f AprilTag used for detection of CART2 position at Datavision s.r.o.
   \par}\vtop{%
      \clabel[fig-camera]{Camera used for AprilTag detection Niceboy Stream Pro}
      \picw=4cm \cinspic imgs/experiments/camera.jpg
      \caption/f Camera used for AprilTag detection Niceboy Stream Pro\cite[camera_niceboy].
   \par}}
\endinsert

\secc CART2 platform description and sensors placements
An image of the utilized CART2 platform can be seen in Figure \ref[fig-cart2_close]. The coordinate frame of CART2, called "baselink",  is illustrated in all Figures \ref[fig-cart2_close] and \ref[fig-cart2_sensors]. The CART2 used for various robotic competitions is a differential drive equipped with the ADlink MXE-210 computer.

\midinsert
\centerline {\picw=4cm \inspic imgs/experiments/cart2_up.pdf \hfil\hfil \picw=7cm \inspic imgs/experiments/cart2_side.pdf }\nobreak
\centerline {a) Top view \hfil\hfil b) Side view}\nobreak\medskip
\clabel[fig-cart2_close]{A photo of the utilized CART2 platform}
\caption/f A photo of the utilized CART2 platform.
\endinsert

The onboard sensor placements are illustrated in Figure \ref[fig-cart2_sensors]. The vehicle uses a Maxon brushless DC motor (Maxon EC-imotor) controlled by Maxon EPOS4 control units.

\midinsert
\line{\hsize=.5\hsize \vtop{%
      \clabel[fig-cart2_sensors]{A photo of sensors placements on CART2}
      \picw=6cm  \cinspic imgs/experiments/cart2_whole.pdf
      \caption/f A photo of sensors placements on CART2.
   \par}\vtop{%
      \clabel[fig-cart2_coordinates]{Illustration of coordinate systems at CART2}
      \picw=\hsize \cinspic imgs/experiments/cart2_coordinates.pdf
      \caption/f Illustration of coordinate systems at CART2.
   \par}}
\endinsert

The UWB tag is mounted on a wooden stick approximately one meter above the CART2 platform to reduce any reflections of the UWB waves from surfaces and the negative influence of any metal parts. The AprilTag is mounted below the UWB tag not to become a barrier in the UWB wave. When the AprilTag is mounted on top of the CART2 platform, the wooden stick obstructs the camera detection of AprilTag. Thus, I decided to mount it as high as possible to reduce any of these situations from happening.

{\bf Coordinate systems of CART2 and sensors placements} are illustrated in Figure \ref[fig-cart2_coordinates] and are described in Table \ref[cart2_frames]. Note, that the world, the camera and the uwb_base frames are fixed with environment and explained in Section \ref[sec-experiments_bu04].

\midinsert \clabel[cart2_frames]{Summary of transforms for experiments with CART2}
\ctable{lrr}{
\hfil Transform & Translation [x, y, z] in [m] & Rotation in quaternion [x, y, z, w] \crl \tskip4pt
uwb_tag to baselink & [0.0, 0.0, 1.192] & [0.0, 0.0, 0.0, 1.0] \cr
april_tag to baselink & [0.0, -0.171, 1.071] & [0.707107, -0.707107, 0.0, 0.0] \cr
imu to baselink & [0.0, 0.0, 0.05] & [0.0, 0.0, -0.7071068, 0.7071068] \cr
}
\caption/t Summary of transforms for experiments with CART2.
\endinsert

\label[sec-experiments_bu04]
\sec Experiments in the lab at Datavision s.r.o.
\secc Experimental lab description
	For the first experiments, I created an experimental setup at Datavision s.r.o. with the global reference given by the camera detection from the AprilTag mounted on the CART2 platform. The address of the building is Ukrajinská 1487/2a, 101 00 Prague 10 - Vršovice. The dimensions of the room for the experiments are approximately 4 x 6 [m], and the camera view area is approximately 2.5 x 4.5 [m]. The setup can be seen in Figure \ref[fig-bu04].
	
\midinsert
    \picw=\hsize \cinspic imgs/experiments/bu04_coordinates.pdf
    \clabel[fig-bu04]{Experimental setup at Datavision s.r.o.}
    \caption/f Experimental setup at Datavision s.r.o..
\endinsert

Firstly, UWB anchors need to be mounted and measure their poses according to the world coordinate system. These poses need to be set in the mobile application for the configuration of the UWB network. The world frame coincides with the uwb_base frame in rotation and only differs in z coordinate in translation.

\midinsert \clabel[bu04_frames]{Summary of transforms for experiments setup at Datavision s.r.o.}
\ctable{lrr}{
\hfil Transform & Translation [x, y, z] in [m] & Rotation in quaternion [x, y, z, w] \crl \tskip4pt
world to uwb_base & [0.0, 0.0, 2.58] & [0.0, 0.0, 0.0, 1.0] \cr
world to camera & [1.089, 2.024, 2.625] & [0.707107, -0.707107, 0.0, 0.0] \cr
}
\caption/t Summary of transforms for experiments setup at Datavision s.r.o.
\endinsert

Secondly, the camera needs to be mounted and measure its position according to the world coordinate system. The transform between the world and the camera is not trivial. Because of that, I decided to use the transformation mentioned in Table \ref[bu04_frames] and then computed a camera homography according to a few positions measured by camera detection and also by hand. The homography defines the transformation between a planar surface (ground) and a camera image plane. The camera homography is then applied to the AprilTag detection pose, which is considered a global reference. These frames are illustrated in \ref[fig-cart2_coordinates].
	
\secc Description of experiments
	The area for experiments is not big, but I decided that these experiments serve as a proof of concept of the localization idea.
	
\midinsert
\centerline {\picw=4.75cm \inspic imgs/experiments/trajectories/stat_and_rot_move.pdf \hfil\hfil \picw=5cm \inspic imgs/experiments/trajectories/xy_move.pdf }\nobreak
\centerline {a) Stationary and rotation tests \hfil\hfil b) X and Y tests}\nobreak\medskip
\centerline {\picw=5cm \inspic imgs/experiments/trajectories/rectangle.pdf \hfil\hfil \picw=4.7cm \inspic imgs/experiments/trajectories/infinity.pdf }\nobreak
\centerline {c) Rectangle test \hfil\hfil d) Infinity test}\nobreak\medskip
\clabel[fig-bu04-traj]{Trajectories for experiments at Datavision s.r.o.}
\caption/f Trajectories for experiments at Datavision s.r.o..
\endinsert
	
Experiments can be divided based on six simple trajectories of CART2 into
 \begitems
 	* stationary test (stationary test),
 	* rotation above 360 degrees in one direction at one place test (rotation test),
 	* moving in x direction (x test),
 	* moving in y direction (y test),
 	* moving in rectangle shape (rectangle test)
 	* and moving in infinity shape (infinity test).
 \enditems
 I picked a specific initial pose for each testing trajectory to repeatedly make these tests in very similar conditions. Also, rotation, rectangle, and infinity tests ran several times in a row without start/stop of the system to see how it behaves in a "long" term.
 
 Trajectories of these tests and the starting and ending positions of CART2 performing the movement are illustrated in Figure \ref[fig-bu04-traj]. CART2 platform was controlled via keyboard and joystick during experiments. The controlling via keyboard has the benefit of control velocity in each direction easily. Thus I used it for constant speed during simple moving in single-axis and simple rotation around a single axis. With that, I controlled all tests, except the infinity test, where the movement is complex. During the rectangle test, CART2 drove forward (moving in single-axis) then stopped and turned at one place above 90 degrees (rotating above single axis). I controlled the CART2 platform with the joystick in infinity shape movement and tried to move similar velocities as in constant movements. The summary of velocities during experiments is given in Table \ref[bu04_traj_vel].
 
\midinsert \clabel[bu04_traj_vel]{Velocities during experiments in Datavision s.r.o.}
\ctable{lrrr}{
\hfil Test & Description & Speed [m/s] & Turn [rad/s] \crl \tskip4pt
stationary & Constant velocities & 0.0 & 0.0 \cr
rotation & Constant velocities & 0.0 & 0.1094 \cr
x & Constant velocities & 0.0750 & 0.0 \cr
y & Constant velocities & 0.0750 & 0.0 \cr
rectangle & Constant velocities & 0.0750 or 0.0 & 0.0 or 0.1094 \cr
infinity test & non constant (control via joystick) & approx. 0.0750 & approx. 0.1094 \cr
}
\caption/t Velocities during experiments in Datavision s.r.o..
\endinsert
  
\secc Evaluation of experiments
TODO

\sec Experiments at CIIRC
\secc Experiments at CIIRC description
\midinsert
\centerline {\picw=0.85\hsize \inspic imgs/experiments/ciirc1.pdf }\nobreak
\centerline {a) View from camera 1}\nobreak\medskip
\centerline {\picw=0.85\hsize \inspic imgs/experiments/ciirc2.pdf }\nobreak
\centerline {b) View from camera 2}\nobreak\medskip
\clabel[fig-ciirc]{Experimental setup at Intelligent and mobile robotics lab (CIIRC)}
\caption/f Experimental setup at Intelligent and mobile robotics lab (CIIRC).
\endinsert
The next experimental environment was placed at the Czech Technical University in Prague – Czech Institute of Informatics, Robotics, and Cybernetics (CIIRC) in the Intelligent and mobile robotics lab. The localization system was evaluated using the VICON external camera localization system \cite[vicon]. The VICON defines the world frame. UWB localization is installed with six anchors and their positions are defined in a world frame. The setup is illustrated in Figure \ref[fig-ciirc]. The benefit of this environment is the size of the laboratory. The minimal distance between two anchors is larger than 3,5 meters; thus, the accuracy of the UWB localization should be increased.

Experiments here are more complex because they are longer both in terms of duration or distance. These experiments promised to evaluate possible drift caused by using two relative localizations based on IMU and odometry. As I already mentioned in previous chapters, localization is based on pure IMU drift with time and odometry drift with traveled distance. Performed experiments can be divided based on three simple trajectories of CART2 into
 \begitems
 	* moving in the shape of a rectangle (rectangle test),
 	* moving in the shape of a infinity symbol (infinity test)
 	* and moving in the shape of a bean (bean test).
 \enditems
 I picked a specific initial pose for each testing trajectory to repeatedly make these tests in very similar conditions. Also, the rotation, the rectangle and the infinity tests ran several times in a row without starting or stopping the system to see how it behaves under longer time durations.

\secc Evaluation of experiments at CIIRC
TODO


\midinsert
\centerline {\picw=8cm \inspic imgs/experiments/results/ciirc/rectangle/position_xy.png \hfil\hfil \picw=8cm \inspic imgs/experiments/results/ciirc/rectangle/orientation_yaw.png }\nobreak
\centerline {a) Position xy \hfil\hfil b) Angle yaw in time}\nobreak\medskip
\centerline {\picw=8cm \inspic imgs/experiments/results/ciirc/rectangle/position_x.png \hfil\hfil \picw=8cm \inspic imgs/experiments/results/ciirc/rectangle/position_y.png }\nobreak
\centerline {c) Position x in time \hfil\hfil d) Position y in time}\nobreak\medskip
\clabel[fig-ciirc-rectangle]{Results of rectangle test at CIIRC with Vicon reference.}
\caption/f Results of rectangle test at CIIRC with Vicon reference.
\endinsert

\midinsert
\centerline {\picw=8cm \inspic imgs/experiments/results/ciirc/infinity/position_xy.png \hfil\hfil \picw=8cm \inspic imgs/experiments/results/ciirc/infinity/orientation_yaw.png }\nobreak
\centerline {a) Position xy \hfil\hfil b) Angle yaw in time}\nobreak\medskip
\centerline {\picw=8cm \inspic imgs/experiments/results/ciirc/infinity/position_x.png \hfil\hfil \picw=8cm \inspic imgs/experiments/results/ciirc/infinity/position_y.png }\nobreak
\centerline {c) Position x in time \hfil\hfil d) Position y in time}\nobreak\medskip
\clabel[fig-ciirc-infinity]{Results of infinity test at CIIRC with Vicon reference}
\caption/f Results of infinity test at CIIRC with Vicon reference.
\endinsert


