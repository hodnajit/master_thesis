\chap Experiments
\sec Used hardware description
\secc Sensors specification
The onboard sensors of interest are Inertial Measurement Unit, ultra-wideband localization tag, april tag and encoders on motors.

{\bf Inertial measurement unit} used during experiments is Epson M-G365PDF1 (loaner sample). Epson M-G365 is used in a number of various applications ranging from stabilization systems (as camera gimbal) or navigation systems.

%\midinsert
%    \picw=0.3\hsize \cinspic imgs/experiments/g365.jpg
%    \clabel[fig-epson-g365]{IMU used in experiments Epson M-G365PDF1}
%    \caption/f IMU used in experiments Epson M-G365PDF1\cite[epson_g365]
%\endinsert

\midinsert
\line{\hsize=.5\hsize \vtop{%
      \clabel[fig-epson-g365]{IMU used in experiments Epson M-G365PDF1}
      \picw=0.5\hsize \cinspic imgs/experiments/g365.jpg
      \caption/f IMU used in experiments Epson M-G365PDF1.
   \par}\vtop{%
      \clabel[fig-epson-avar]{Experimental setup for static data acquisition of Epson M-G365PDF1}
      \picw=\hsize \cinspic imgs/experiments/avar_epson_setup.png
      \caption/f Experimental setup for static data acquisition of Epson M-G365PDF1.
   \par}}
\endinsert

The IMU has six degrees of freedom and measures angular rates and linear accelerations in three axis. It is factory calibrated and the calibration data are stored in the memory of the unit. Technical specifications of Epson M-G365 can be found at \fnote{Datasheet of Epson M-G365 \url{https://global.epson.com/products_and_drivers/sensing_system/download_hidden/pdf/m-g365pd_datasheet_e_rev20201007.pdf}}, while summary is included in Table \ref[spec_epson]

\midinsert \clabel[spec_epson]{Technical specifications of Epson M-G365PDF1}
\ctable{lr}{
\hfil
Specification & Value \crl \tskip4pt
Triple gyroscopes & $\pm$ 450 °/sec \cr
Gyroscopes bias instability & 1.2 °/hr \cr
Gyroscopes initial bias error & 0.1 °/s \cr
Angular random walk & 0.08 °/√hr \cr
Tri-axis accelerometers & $\pm$ 10 G \cr
Accelerometers bias instability & 16 $\mu$G \cr
Accelerometers initial bias error & 3 mG \cr
Velocity random walk & 0.033 (m/s)/√hr \cr
}
\caption/t Technical specifications of Epson M-G365PDF1\cite[epson_g365].
\endinsert

As I already mentioned in Chapter \ref[chap-sec-avar], the AVAR analysis of IMU sensors can give us a brief overview of IMU's specification. Experimental setup for static data acquisition can be seen in Figure \ref[fig-epson-avar]. The sensor is mounted on two sponge and fixed with cartboard. The static data were recorded for 48 hours at frequency of 30.0 Hz. The experiment took place at a village without any subway, trams or trains to reduce external vibrations on measurement (to reduce potential outliers) and at standard room temperature (about 23 °C).

For AVAR computation I used a python library named AllanTools, for purposes of this thesis the overlapping Allan deviation function is used.

\picw=.6\hsize
\centerline {\inspic imgs/experiments/epson_accelerometers.pdf \hfil \inspic imgs/experiments/epson_gyroscopes.pdf }\nobreak
\centerline {a)\hfil b)}\nobreak\medskip
\clabel[fig-epson-avar2]{Overlapping Allan variance plot for Epson M-G365PDF1.}
\caption/f Overlapping Allan variance plot for Epson M-G365PDF1.

As you can see in the Figure \ref[fig-epson-avar2], the external reset of integration of IMU measurement must be performed at 0.01 Hz at the latest. After that period, non gaussian noise of accelerometers comes into account. The UWB localization system works at 10 Hz, that should be enough for the reseting.

For the final experiments, the IMU is setup to publish delta angle and delta velocity at 100 Hz with moving average filter with tap 64. Because of that, higher frequencies than 10 Hz are filtered as can be seen in Figure \ref[fig-epson-filter].

\midinsert
    \picw=0.9\hsize \cinspic imgs/experiments/epson_filter.png
    \clabel[fig-epson-filter]{Moving average filter characteristics for Epson M-G365PDF1.}
    \caption/f Moving average filter characteristics for Epson M-G365PDF1\cite[epson_g365].
\endinsert

{\bf UWB localization system} is povided by Qorvo's MDEK1001 ultra-wideband development kit. This kit includes 12 DWM1001-DEV development boards in plastic enclosures (see Figure \ref[fig-uwb-dwm]).

\midinsert
    \picw=0.4\hsize \cinspic imgs/experiments/mdek1001.jpg
    \clabel[fig-uwb-dwm]{DWM1001-DEV development boards.}
    \caption/f DWM1001-DEV development boards\cite[mdek_1001].
\endinsert

Each can be configured as an anchor, tag or bridge node. The system is installed with 6 anchors fixedly mounted and one tag which can be moved. The anchors are higher than tag. According to the initiator anchor, the tag’s position gives the coordinates frame’s origin. The set up of the system can be seen in Figure \ref[fig-uwb-anchors_tags].

\midinsert
    \picw=0.8\hsize \cinspic imgs/experiments/uwb_anchors_tags.png
    \clabel[fig-uwb-anchors_tags]{Positioning of anchors and tags.}
    \caption/f Positioning of anchors and tags\cite[mdek_1001].
\endinsert

The anchors should be mounted at the same hight and higher than the operating area of the moving tag. They also should not be mounted close to any metal to get the best accuracy. The configuration of the whole network is done via their mobile application and positions of the anchors are estimated manually.

Technical specification of MDEK1001 and DWM1001-DEV can be found in documents \cite[mdek_1001], the summary is listed in Table \ref[spec_mdek1001].

\midinsert \clabel[spec_mdek1001]{System performance of MDEK1001}
\ctable{lr}{
\hfil
Specification & Value \crl \tskip4pt
Maximum tag location rate & 10 Hz \cr
X-Y location accuracy & < 10 cm \cr
Point to point range & up to 60 m in line of sight conditions \cr
Scheme range & 25 - 30 m between anchors\cr
}
\caption/t System performance of MDEK1001\cite[mdek_1001].
\endinsert

{\bf TODO The odometry} is computed according to measurement of encoders in Maxon EPOS4 positioning controllers for Maxon brushless DC motors.

{\bf AprilTag} serves as a global reference of pose of the CART2 platform. The AprilTag detection software computes the precise 3D position, orientation, and identity of the tags relative to the camera\cite[april_tag]. This tag is similar to QR codes (a type of two-dimensional bar code), but it encodes smaller data payloads (between 4-12 bits) and it can be detected more robustly.

{\bf Camera} used for detection of AprilTag is Niceboy Stream Pro with Full HD (1920 x 1080) resolution, 30 FPS, 90 ° field of vision and f/1.8 lens aperture\cite[camera_niceboy].

\midinsert
\line{\hsize=.5\hsize \vtop{%
      \clabel[fig-apriltag]{AprilTag used for detection of CART2 position}
      \picw=4cm \cinspic imgs/experiments/apriltag.png
      \caption/f AprilTag used for detection of CART2 position.
   \par}\vtop{%
      \clabel[fig-camera]{Camera used for AprilTag detection Niceboy Stream Pro}
      \picw=4cm \cinspic imgs/experiments/camera.jpg
      \caption/f Camera used for AprilTag detection Niceboy Stream Pro\cite[camera_niceboy].
   \par}}
\endinsert

\secc CART2 platform description and sensors placements
An image of the utilized CART2 platform can be seen in Figure \ref[fig-cart2_close]. The coordinate frame of CART2 called "baselink"  is illustrated in all Figures \ref[fig-cart2_close] and \ref[fig-cart2_sensors]. The CART2 used for a various robotic competitions is a differential drive and is equipped with the ADlink MXE-210 computer.

\midinsert
\centerline {\picw=4cm \inspic imgs/experiments/cart2_up.pdf \hfil\hfil \picw=7cm \inspic imgs/experiments/cart2_side.pdf }\nobreak
\centerline {a) Top view \hfil\hfil b) Side view}\nobreak\medskip
\clabel[fig-cart2_close]{A photo of the utilized CART2 platform}
\caption/f A photo of the utilized CART2 platform.
\endinsert

The onboard sensors placements are illustrated in Figure \ref[fig-cart2_sensors]. The vehicle uses Maxon brushless DC motor, which is controlled by Maxon EPOS4 control units.

\midinsert
\line{\hsize=.5\hsize \vtop{%
      \clabel[fig-cart2_sensors]{A photo of sensors placements on CART2}
      \picw=6cm  \cinspic imgs/experiments/cart2_whole.pdf
      \caption/f A photo of sensors placements on CART2.
   \par}\vtop{%
      \clabel[fig-cart2_coordinates]{Illustration of coordinate systems at CART2}
      \picw=\hsize \cinspic imgs/experiments/cart2_coordinates.pdf
      \caption/f Illustration of coordinate systems at CART2.
   \par}}
\endinsert

The UWB tag is mounted on a wooden stick aproximately one meter upon the CART2 platform to reduce reflections of UWB wave from surfaces and negative influence of any metal part. The AprilTag is mounted bellow the UWB tag to not become a barrier in UWB wave. When the AprilTag was mounted on top of the CART2 platform, the wooden stick obstructs the camera detection of AprilTag in some position of CART2 in a room, thus I decided to mount it as high as possible to reduce these situations.

{\bf Coordinate systems of CART2 and sensors placements} are illustrated in Figure \ref[fig-cart2_coordinates] and described in Table \ref[cart2_frames].

\midinsert \clabel[cart2_frames]{Summary of transforms for experiments with CART2}
\ctable{lrr}{
\hfil Transform & Translation [x, y, z] in [m] & Rotation in quaternion [x, y, z, w] \crl \tskip4pt
world to uwb_base & [0.0, 0.0, 2.58] & [0.0, 0.0, 0.0, 1.0] \cr
uwb_base to baselink & [0.0, 0.0, 1.32] & [0.0, 0.0, 0.0, 1.0] \cr
april_tag to baselink & [0.0, -0.171, 1.12] & [0.707107, -0.707107, 0.0, 0.0] \cr
imu to baselink & [0.0, 0.0, 0.05] & [0.0, 0.0, -0.7071068, 0.7071068] \cr
}
\caption/t Summary of transforms for experiments with CART2.
\endinsert
	
\sec Experiments in experimental room at Datavision s. r. o.
\secc Experimental room at Datavision s. r. o. description
	For first experiments I created an experimental setup at Datavision s. r. o. with global reference given by camera detection of AprilTag mounted at CART2 platform. The address of the building is Ukrajinská 1487/2a, 101 00 Prague 10 - Vršovice. Dimensions of the room for experiments are approximately 4 x 6 [m] and the camera view area is approximately 2.5 x 4.5 [m]. The setup can be seen in Figure \ref[fig-bu04].
	
\midinsert
    \picw=\hsize \cinspic imgs/experiments/bu04_coordinates.pdf
    \clabel[fig-bu04]{Experimental setup at Datavision s. r. o.}
    \caption/f Experimental setup at Datavision s. r. o..
\endinsert

Firstly, UWB anchors need to be mounted and measured their poses according to the world coordinate system. These poses need to be set in the mobile application, where all the uwb network can be set up. The world frame coincidents to uwb_base frame in rotation and only differs in z coordinate in translation.

\midinsert \clabel[bu04_frames]{Summary of transforms for experiments setup at Datavision s. r. o.}
\ctable{lrr}{
\hfil Transform & Translation [x, y, z] in [m] & Rotation in quaternion [x, y, z, w] \crl \tskip4pt
world to uwb_base & [0.0, 0.0, 2.58] & [0.0, 0.0, 0.0, 1.0] \cr
world to camera & [1.089, 2.024, 2.625] & [0.707107, -0.707107, 0.0, 0.0] \cr
}
\caption/t Summary of transforms for experiments setup at Datavision s. r. o..
\endinsert

Secondly, the camera need to be mounted and measured its position according to the world coordinate system. The transform between world and camera is not trivial, because of that I decided to use transformation mentioned in Table \ref[bu04_frames] and then computed a camera homography according to a few positions measured by camera detection and by hand. The homography defines the transformation between a planar surface (ground) and camera image plane. The camera homography is then applied to the AprilTag detection pose and this pose is considered as global reference.
	
\secc Description of experiments
	The area for experiments is not big, but I decided that these experiments serve as a proof of concept, that the proposed localization works.
	
\midinsert
\centerline {\picw=4.75cm \inspic imgs/experiments/trajectories/stat_and_rot_move.pdf \hfil\hfil \picw=5cm \inspic imgs/experiments/trajectories/xy_move.pdf }\nobreak
\centerline {a) Stationary and rotation tests \hfil\hfil b) X and Y tests}\nobreak\medskip
\centerline {\picw=5cm \inspic imgs/experiments/trajectories/rectangle.pdf \hfil\hfil \picw=4.7cm \inspic imgs/experiments/trajectories/infinity.pdf }\nobreak
\centerline {c) Rectangle test \hfil\hfil d) Infinity test}\nobreak\medskip
\clabel[fig-bu04-traj]{Trajectories for experiments at Datavision s. r. o.}
\caption/f Trajectories for experiments at Datavision s. r. o..
\endinsert
	
	I picked six simple trajectories for experiments and tests can be divided according to these trajectories as
 \begitems
 	* stationary test (stationary test),
 	* rotation above 360 degrees in one direction at one place test (rotation test),
 	* moving in x direction test (x test),
 	* moving in y direction test (y test),
 	* moving in rectangle shape test (rectangle test),
 	* and moving in infinity shape (infinity test).
 \enditems
 For each testing trajectories I picked specific initial pose to be possible to make these tests repeatively in very similar conditions. Also rotation, rectangle and infinity tests ran several times in a row without start/stop of the system to see, how it behaves in a "long" term.
 
 Trajectories of these tests as well as the starting and ending positions of CART2 performing the movement are illustrated in Figure \ref[fig-bu04-traj]. CART2 platform was controlled via keyboard and joystick during experiments. The controlling via keyboard has a benefit of control velocity in each direction easily, thus I used it for constant velocity during simple moving in single axis and simple rotation around single axis. With that I controlled all tests, except infinity test, where the movement is more complicated. During rectangle test, CART2 drove forward (moving in single axis) then stopped and turned at one place above 90 degrees (rotating above single axis). In infinity movement I drove the CART2 platform with joystick and tried to move similar velocities as in constant movements. The summary of velocities during experiments is given in Table \ref[bu04_traj_vel].
 
\midinsert \clabel[bu04_traj_vel]{Velocities during experiments in Datavision s. r. o.}
\ctable{lrrr}{
\hfil Test & Description & Speed [m/s] & Turn [rad/s] \crl \tskip4pt
stationary & Constant velocities & 0.0 & 0.0 \cr
rotation & Constant velocities & 0.0 & 0.1094 \cr
x & Constant velocities & 0.0750 & 0.0 \cr
y & Constant velocities & 0.0750 & 0.0 \cr
rectangle & Constant velocities & 0.0750 or 0.0 & 0.0 or 0.1094 \cr
infinity test & non constant (control via joystick) & approx. 0.0750 & approx. 0.1094 \cr
}
\caption/t Velocities during experiments in Datavision s. r. o..
\endinsert
  
\secc Evaluation of experiments at Datavision s. r. o.	
\sec Experiment at CTU CIIRC
\secc Experimental room at CIIRC description
\secc Description of experiments
\secc Evaluation of experiments at CIIRC

