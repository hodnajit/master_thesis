\chap Indoor localization methods

\sec Absolute and relative localization
In the field of localization systems, plenty of technologies have been investigated. These technologies can be divided into two significant subgroups according to a process, how the location is described. {\bf Absolute} localization refers to the system, where the position is given in some coordinate system, and in {\bf relative} localizations, the location is described with respect to other locations or landmarks.
Typically, in the absolute localization system, the position should not significantly drift over time but can change in discrete jumps. That means that this system is not continuous but serves good as a long-term global reference. In the relative localization system, the position drifts over time but does not suffer for discrete jumps, as it is continuous.

It can be seen that the best approach for designing the localization system is to combine both principles to counter their negative aspects. The combination is given by state estimations algorithms briefly described in Chapter \ref[chap-state-est]. But before that, the different technologies for position estimation are briefly introduced in the next section.

\sec A brief overview of localization technologies
- lokalizace z kamery (to chci rozebrat kvuli referenci - apriltag a vicon
- laserova lokalizace
- lokalizace v prumyslu nyni
\sec State of the art in fusion UWB a IMU
- jake jsou trendy ve fuzi uwb a neco?




Jednim z nejvetsich problemu indoor lokalizace je nemoznost pouzit absolutni lokalizaci pomoci GNSS, jelikoz ted pro sve spravne fungovani potrebuje primy pohled na oblohu. Dnes se tedy indoor lokalizace robotu resi ruznymi druhy technologii, vetsinou i ve spravne fuzi. Pristupy lokalizace se daji delit bud podle pouzite technologie nebo 
Obecne se da lokalizace rozdelit na dve kategorie a to lokalizace absolutni a lokalizace relativni.
Vyznam absolutni a relativni lokalizace.
Dnes se 
