\label[chap-state-est]
\chap State estimations algorithms for localization
Localisation is the problem of estimating a robot’s coordinates in an external reference frame from sensor data. In particular, probabilistic approaches are typically more robust in the face of sensor limitations, sensor noise or environment dynamics\cite[probabilistic_robotics]. Moreover, they often scale much better to complex and unstructured environments, where the ability to handle uncertainty is of even greater importance\cite[probabilistic_robotics].

For that reason, only probabilistic methods for state estimation are described in the next chapter. The basis of each technique briefly described in this thesis is Bayes filters\cite[probabilistic_robotics].
 The first section discusses the difference between Kalman and Particle filter for state estimation. The following section introduces advanced concepts derived from the Kalman filter algorithm as extended Kalman filter and unscented Kalman filter. The third and final section goes deeply into the Error state extended Kalman filter and introduces the benefits of using it.

\sec Kalman and particle filter
Both Kalman and particle filter are the first implementations of Bayes filter in the continuous time\cite[probabilistic_robotics], and in both filters, the state is represented by belief, which corresponds to a distribution. It is a multivariate normal distribution for the Kalman filter, but for particle filter, the distribution is represented by all particles\cite[probabilistic_robotics, kalman_vs_particle].

Both algorithms work with a prediction and correction step, which works with the system and sensor model, respectively. Firstly, it predicts the state based on the internal system model, and secondly, it corrects itself by external measurements and sensor model. Kalman and particle filter algorithms for localization are well described in a referenced Probabilistic robotics\cite[probabilistic_robotics].

There are a few limitations for both algorithms. For the Kalman filter, the state transitions and measurements need to be linear with added Gaussian noise, and the initial state must have normal distribution\cite[probabilistic_robotics].

There is no such requirement for linearity for the particle filter, and it works fine with nonlinear or multi-modal systems too\cite[probabilistic_robotics]. But the algorithm can be more computation demanding as a high number of particles needs to be generated in each sample time for a good estimation\cite[probabilistic_robotics].

I decided to use an algorithm based on the Kalman filter for several reasons.
First, the localization system will be used for the navigation of vehicles. For these kinds of tasks, the update rate has to be relatively high.
Second, the state transition and measurements are approximately linear.
And third, these algorithms are typically used in the fusion of IMU and GNSS, as I already mentioned in the chapter \ref[chap-introduction].

\sec Algorithms based on Kalman filter
There have been many modifications and extensions of the standard Kalman filter since the 1950s when the filter was invented because the assumptions of the linear system and sensor model with added Gaussian noise are rarely fulfilled in practice\cite[probabilistic_robotics]. In that case, the state transition or sensor model are described by nonlinear functions.

There exist many techniques for linearizing nonlinear functions. The most popular tool, called Extended Kalman filter, use (first-order) Taylor expansion\cite[probabilistic_robotics]. This approximation has its limitations, which correspond to a degree of nonlinearity of the functions and a degree of uncertainty. The higher these degrees are, the further the approximation deviates from true belief. In general, the Extended Kalman filter has its benefits in simplicity, optimality and robustness\cite[probabilistic_robotics], but in practice, it is reliable for the system, which is almost linear in one-time step\cite[ukf].

The Unscented Kalman filter is a tool that appears superior to the EKF linearization\cite[probabilistic_robotics, ukf]. The linearization there is given by carefully selected sample points from nonlinear functions. Also, this approach does not assume that the distribution of noise source is Gaussian\cite[ukf].

The degree of nonlinearity of the system is critical for the state estimation by algorithms based on the Kalman filter. Thus, the relatively recent but promising tool Error state Extended Kalman filter was introduced. In this concept, the error of the state is estimated, as it is more likely correctly modelled by a linear function\cite[error_state_kalman, quaternion_kinematics, farrell_aided]. 

\sec Error state Extended Kalman filter
Error state Extended Kalman filter belongs to a group of Indirect Kalman filters, because it does not estimate the state itself, but the error of the state\cite[error_state_kalman].

The main idea behind is that the true state, which should be the output of the system, is computed as a suitable composition of nominal state and the error state
$$
	x_t = x_n \bigoplus \delta x
	\eqmark
$$
where
\begitems
* $\bigoplus$ is a suitable composition as linear sum or matrix product,
* $x_t$ is the true state,
* $x_n$ is the nominal state
* and $\delta x$ is the error state.
\enditems

The nominal state is consider as large signal, which can be integrated in its nonlinear form and the error state as small signal, that is nearly linear function, ideal for Extended Kalman filtering\cite[quaternion_kinematics].

The algorithm can be illustrated in following set of equations in prediction, correction and injection steps.
In {\bf the prediction step} the nominal state and its covariance is estimated by following equation\cite[error_state_kalman, quaternion_kinematics]
$$
\eqalign{
	x_{n_{k}} &= f(x_{t_{k-1}}, u_k) \cr
	P_{n_{k}} &= F P_{t_{k-1}} F^T + B Q B^T \cr
}
	\eqmark
$$
where
\begitems
* $f(x_{t_{k-1}}$ is the nonlinear function described the current nominal state of the system based on previous state and current inputs,
* $x_{n_{k}}$ is the nominal state in time step $k$,
* $x_{t_{k-1}}$ is the true state in time step $k-1$,
* $u$ is the inputs of the system,
* $P_{n_{k}}$ is the nominal state covariance matrix (also called process noise covariance),
* $F$ 
* $B$ 
* and $Q$ 
\enditems

TODO: promysli si, jak tyhle rovnice rozumne zapsat... Otazka je, zda se muze udelat jednoducha demonstrace jako v clanku https://notanymike.github.io/Error-State-Extended-Kalman-Filter/ a nebo nee :(












