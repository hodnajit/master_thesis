\input ctustyle3
\worktype [M/EN]
\faculty {F3}
\department {Department of Control Engineering}
\title {Indoor localization system for automated vehicles based on Ultra-Wideband technology}
\author {Bc. Jitka Hodná}
\studyinfo {Cybernetics and robotics}
\workinfo {
\picw=0.1\hsize % obrázek na šířku sazby
\linspic imgs/datavision.png
Datavision s. r. o.}
\date {May 2021}
\supervisor {Ing. Tomáš Novák}
\abstractEN {This thesis proposes a novel approach in indoor real-time localization of autonomous ground vehicles and aims to ensure its accuracy, repeatability and reliability in the long term.

The system is based on Ultra Wide-Band technology, an Inertial Measurement Unit and odometry. An integration scheme using an Error state Extended Kalman filter in a closed-loop fashion is employed for the realization of this approach.

Evaluation is based on real data obtained from various testing scenarios in two different environments using the proposed aided inertial navigation system integrated on the CART2 robotic platform. The results show that the system can achieve a positioning precision within 10 cm and a few degrees in attitude. The system meets the requirements for practical conditions and seems viable for indoor localization with high accuracy and precision demands.}
\abstractCZ {Tato práce představuje inovativní přístup v oblasti interiérové lokalizace pro autonomní pozemní vozidla se zaměřením na její dlouhodobou přesnost a spolehlivost.

Systém je založený na fúzi ultra wide-band technologie, inerciální měřící jednotky a odometrie. Realizace je uskutečněna jako inerciální navigační systém zpětnovazebně řízený rozšířeným Kalmanovovým filtrem
s oddylkovým modelem.

Navržený systém je implementován na robotickou platformu CART2 a experimentálně ověřen na několika testovacích scénárií ve dvou různých vnitřních prostředích. Výsledky ukazují, že systém dosahuje přesnosti v pozici až 10 cm a v orientaci jednotek stupňů. Systém splňuje požadavky pro jeho využití v praktických nasazeních, kde je vyžadována vysoká přesnost a spolehlivost.}
\titleCZ {Interiérový lokalizační systém pro autonomní prostředky s využitím technologie Ultra-Wideband}
\keywordsEN {ultra-wideband, imu, ins, localization, indoor, kalman-filter, ekf, indoor localization, error state extended kalman filter}
\keywordsCZ {ultra-wideband, imu, ins, localizace, vnitřní prostředí, ekf, rozšířený kalmanův filtr, odchylkový model rozšířeného kalmanova filtru}
\thanks {I want to thank my supervisor {\it Ing. Tomáš Novák} and the company {\it Datavision s.r.o.} for the possibility to work on such an exciting topic as the mobile robotic is.

I would also like to thank the {\it Technology Agency of the Czech Republic} for financial support of the project Guidance and Localization upgrade creating Autonomous Mobile Robots under the {\it TREND Program FW03010020}.

Thank you, my beloved alma mater {\it Czech Technical University in Prague, Faculty of Electrical Engineering}, for gaining knowledge, for the opportunity to meet a lot of interesting personages and for every professor who {\it patiently} and {\it respectfully} shares their expertise with a new generation.

Thanks to {\it Ing. Martin Šipoš}, {\it Ph.D. and doc. Ing. Jan Roháč, Ph.D.} for an introduction to the inertial navigation systems.
Thanks to {\it Ing. Tomáš Báča} for his help and implementation suggestions.
Thanks to my friend {\it Ing. David Zahrádka} and {\it the Intelligent and Mobile Robotics Group} at CTU CIIRC for the possibility to complete experiments with the Vicon reference system.

Besides, I would like to thank my friends and family for their support. Namely, to {\it Lucie Halodová}, for being such a fantastic buddy.}
\declaration {I hereby declare that I wrote the presented thesis on my own and that I cited all the used information sources in compliance with the Methodical instructions about the ethical principles for writing an academic thesis.
\signature

\hfill
Prague, May 21, 2021}
\specification {\picw=\hsize \cinspic imgs/zadani.pdf}

\makefront

\input tex/introduction

% \input tex/indoor_localization

\input tex/sensors

\input tex/state_estimation

\input tex/rex_act_localization

\input tex/experiments

\input tex/conclusion


\input tex/attachments


\bye
